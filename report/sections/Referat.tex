\def\npages{52}
\def\ntables{2}
\def\nris{4}
\def\ncite{14}



\begin{center}
  \fontsize{16pt}{0em}\textbf{АННОТАЦИЯ}
\end{center}

Крачковский Д.Я. \ Вычислительные алгоритмы положительной матричной факторизации: Дипломная работа / Минск: БГУ, \the\year. -\npages\ c.
\\

Рассматривается применение вычислительных алгоритмов неотрицательной матричной факторизации для решения задачи интеллектуального анализа текста.
Реализовано 4 метода решения задачи НМФ и проведено их экспериментальное сравнение а также они были применены к задаче выделения ключевых слов и предложений в текстах.



\begin{center}
  \fontsize{16pt}{0em}\textbf{АНАТАЦЫЯ}
\end{center}

Крачкоўскі Д.Я. \ Вылічальныя алгарытмы неадмоўнай матрычнай фактaрызацыi: Дыпломная праца / Минск: БДУ, \the\year. -\npages\ c.
\\

Разглядаецца прымяненне вылічальных алгарытмаў неадмоўнай матрычнай фактарызацыi для вырашэння задачы інтэлектуальнага аналізу тэксту.
Рэалізавана 4 метады рашэння задачы НМФ і праведзена іх эксперыментальнае параўнанне, а таксама яны былі ўжытыя да задачы вылучэння ключавых слоў і сзазаў у тэкстах.


\begin{center}
  \fontsize{16pt}{0em}\textbf{ANNOTATION}
\end{center}

Krachkouski D.Y. \ Computational algorithms of nonnegative matrix factorization: Diploma work / Minsk: BSU, \the\year. -\npages\ p.
\\

The use of computational algorithms for nonnegative matrix factorization to solve the problem of text mining is considered.
4 methods of solving the NFM problem were implemented and they were experimentally compared, and they were also applied to the problem of text keywords and sentences extraction.

\newpage



\begin{center}
  \fontsize{16pt}{0em}\textbf{РЕФЕРАТ}
\end{center}

Дипломная работа, \npages\ страниц, \ntables\ таблиц, \nris\ рисунков, \ncite\ источника.

Ключевые слова:
НЕОТРИЦАТЕЛЬНАЯ МАТРИЧНАЯ ФАКТОРИЗАЦИЯ (НМФ),
АППРОКСИМАЦИЯ НИЗКОГО РАНГА,
ИНТЕЛЛЕКТУАЛЬНЫЙ АНАЛИЗ ДАННЫХ,
ЧИСЛЕННЫЕ МЕТОДЫ, ЛИНЕЙНАЯ АЛГЕБРА,
РАЗРЕЖЕННЫЕ МАТРИЦЫ,
ТЕРМ-ДОКУМЕНТНАЯ МАТРИЦА,
МЕТОД НАИМЕНЬШИХ КВАДРАТОВ.

\textit{Объект исследования} - вычислительные алгоритмы неотрицательной матричной факторизации.

% объект исследования, цель работы, методы или методологию
% проведения работы, полученные результаты и их новизну,

\textit{Цель работы} - исследование эффективности алгоритмов неотрицательной матричной факторизации и их применение к задаче интеллектуального анализа текста.

Основным методом проведения работы является вычислительный эксперимент. В ходе работы было программно реализовано 4 метода решения задачи НМФ
(метод мультипликативного обновления и 3 различных модификации метода попеременных наименьших квадратов), проведено их экспериментальное сравнение.
Реализованные алгоритмы были применены к задаче выделения ключевых слов и предложений в текстах.
Результаты исследование показали что наиболее эффективным алгоритмом оказался метод попеременных наименьших квадратов на основе нормальных уравнений.
Метод выделения ключевых слов и предложений на основе НМФ был применён к тексту первой главы данной работы.
Полученные результаты хорошо резюмируют данный текст.

Областью применения являются рекомендательные системы, задачи интеллектуального анализа текстов, поисковые системы, компьютерное зрение,
биоинформатика, спектральный анализ данных и другие.


\newpage


\begin{center}
  \fontsize{16pt}{0em}\textbf{РЭФЕРАТ}
\end{center}

Дыпломная праца, \npages\ старонак, \ntables\ таблiц, \nris\ малюнкаў, \ncite\ крынiц.

Ключавыя словы:
НЕАДМОЎНАЯ МАТРЫЧНАЯ ФАКТАРЫЗАЦЫЯ (НМФ),
АПРАКСІМАЦЫЯ НІЗКАГА РАНГУ,
ІНТЭЛЕКТУАЛЬНЫ АНАЛІЗ ДАДЗЕНЫХ,
ВЫЛІЧАЛЬНЫЯ МЕТАДЫ, ЛІНЕЙНАЯ АЛГЕБРА,
РАЗРЭДЖАНЫЯ МАТРЫЦЫ,
ТЭРМ-ДАКУМЕНТНАЯ МАТРЫЦА,
МЕТАД НАЙМЕНШЫХ КВАДРАТАЎ.

\textit{Аб'ект даследавання} - вылічальныя алгарытмы неадмоўнай матрычнай фактарызацыi.

% объект исследования, цель работы, методы или методологию
% проведения работы, полученные результаты и их новизну,

\textit{Мэта працы} - даследаванне эфектыўнасці алгарытмаў неадмоўнай матрычнай фактарызацыi і іх прымяненне да задачы інтэлектуальнага аналізу тэксту.

Асноўным метадам правядзення працы з'яўляецца вылічальны эксперымент. У ходзе працы было праграмна рэалізавана 4 метады рашэння задачы НМФ
(Метад мультыплікатыўнага абнаўлення і 3 розных мадыфікацыі метаду папераменных найменшых квадратаў), праведзена іх эксперыментальнае параўнанне.
Рэалізаваныя алгарытмы былі ўжытыя да задачы вылучэння ключавых слоў і сказаў у тэкстах.
Вынікі даследаванне паказалі, што найбольш эфектыўным алгарытмам апынуўся метад папераменных найменшых квадратаў на аснове нармальных раўнанняў.
Метад вылучэння ключавых слоў і прапаноў на аснове НМФ быў ужыты да тэксту першага раздзела дадзенай працы.
Атрыманыя вынікі добра рэзюмуюць дадзены тэкст.

Вобласцю ўжывання з'яўляюцца рэкамендацыйныя сістэмы, задачы інтэлектуальнага аналізу тэкстаў, пошукавыя сістэмы, камп'ютэрны зрок,
біяінфарматыка, спектральны аналіз дадзеных і іншыя.




\newpage




\begin{center}
  \fontsize{16pt}{0em}\textbf{ABSTRACT}
\end{center}

Diploma work, \npages\ pages, \ntables\ tables, \nris\ pictures, \ncite\ references.

Key words:
NONNEGATIVE MATRIX FACTORIZATION (NMF),
LOW-RANK APPROXIMATION,
DATA MINING,
NUMERICAL METHODS,
LINEAR ALGEBRA,
SPARSE MATRIX,
TERM-DOCUMENT MATRIX,
LEAST SQUARES METHOD.

\textit{The object} of study is numerical methods of nonnegative matrix factorization.

% объект исследования, цель работы, методы или методологию
% проведения работы, полученные результаты и их новизну,

\textit{The purpose} of the work is to study the efficiency of nonnegative
matrix factorization algorithms and to apply them to a data mining problem.

The main method of study is the computational experiment.
During the study 4 methods of NMF problem solution were implemented
(multiplicative update rule and 3 different modifications of alternating least squares method).
The comparative numerical experiment was performed.

The implemented algorithms were applied to automatic key words and key sentences extraction problem.
The results showed that the most efficient algorithm is alternating least squares based on normal equations.
Key words and sentences extraction algorithm based on NMF was applied to the text of the first chapter of this work.
The results summarise the text well.

The applications areas are recommendation systems, text mining problems, search engines, computer vision,
bioinformatics, spectral data analysis and others.

\newpage
