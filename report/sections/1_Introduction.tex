% Введение:
% - О задаче НМФ
% - в чем трудности
% - где применяется
% - постановка задачи
% - что сделано в работе
% - кратко о полученных результатах

% \newpage

\chapter*{ВВЕДЕНИЕ}
\addcontentsline{toc}{chapter}{Введение}

На сегодняшний день объемы информации растут в геометрической прогрессии.
Ярким примером  скопления обширного количества данных является социальная сеть Instagram переступившая порог в 1 млрд пользователей.
Для того чтобы быстрее реагировать на изменения рынка, получить конкурентные преимущества,
повысить эффективность сервиса нужно получить, обработать и проанализировать огромное количество данных.
Обработка таких больших объемов данных создаёт новые проблемы в отношении представления данных,
упразднения неоднозначности и уменьшения размерности.

Неотрицательная матричная факторизация (НМФ) - это метод уменьшения размерности.
Многие методы уменьшения размерности тесно связаны с приближениями матрицами низкого ранга,
и неотрицательная матричная факторизация является особенной в том,
что искомые матрицы низкого ранга ограничены только неотрицательными элементами.
За последнее десятилетие
неотрицательная матричная факторизация получила огромное внимание и была успешно применена в широком спектре
важных проблем в таких областях, как интеллектуальный анализ текста, компьютерное зрение,
биоинформатика, спектральный анализ данных, разделение слепых источников и многих других.

Во многих ситуациях данные, наблюдаемые из сложных процессов и взаимодействий
представляют собой совокупный результат нескольких взаимосвязанных переменных, действующих вместе.
Когда эти переменные определены с некоторыми неточностями, фактическая информация, содержащаяся в исходных данных,
может перекрываться и быть неоднозначной.
В данном случае модель с упрощённой системой может обеспечить точность около уровня точности исходной системы.
Одна общая идея в различных подходах к устранению шума, уменьшению размерности модели,
восстановлению непротиворечивости и т.д. заключается в замене исходных данных данными меньшей размерности,
полученными с помощью аппроксимации некоторым подпространством.
Эта идея получила название аппроксимации низкого ранга.
Использование аппроксимации низкого ранга выходит на передний план в широком спектре важных задач.
Факторный анализ и метод главных компонент являются двумя из многих классических методов,
используемых для достижения цели сокращения числа переменных и выявления структур среди переменных.

Часто анализируемые данные неотрицательны, что накладывает следующие ограничения на аппроксимирующие данные:
данные более низкого ранга также должны состоять из неотрицательных значений, чтобы избежать противоречий с природой исходных данных.
Классические инструменты не могут гарантировать сохранение неотрицательности.
Таким образом, естественным выбором становится поиск неотрицательных множителей пониженного ранга для аппроксимации данной неотрицательной матрицы данных.

Цель настоящей работы состоит в исследовании эффективности алгоритмов неотрицательной матричной факторизации и их применение к задаче интеллектуального анализа текста.

В ходе работы было программно реализовано 4 метода решения задачи НМФ
(метод мультипликативного обновления и 3 различных модификации метода попеременных наименьших квадратов), проведено их экспериментальное сравнение.
Реализованные алгоритмы были применены к задаче выделения ключевых слов и предложений в текстах.
