% Заключение (еще раз повторить постановку, четко обозначить то, что было сделано и основные полученные результаты исследования)

\newpage
\section{Заключение}


Мультипликативные алгоритмы требуют намного больше итераций, чем альтернативы, такие как алгоритмы попеременных наименьших квадратов, и работа на онду итерацию предельно высока. Каждая итерация требует шесть $O(n^3)$ матрично-матричных умножений полностью плотных матриц и шесть
$O (n^2)$ покомпонентных операций.

Один недостаток мультипликативных алгоритмов состоит в том, что как только элемент в $W$ или $H$ становится 0, он будет оставаться равным 0. Эта блокировка нулевых элементов означает, что, как только алгоритм начинает двигаться по пути к фиксированной точке, даже если это плохая фиксированная точка, он будет продолжать двигаться в том же направлении.

Алгоритмы ALS являются более гибкими, позволяя итеративному процессу уйти с неправильного пути. В зависимости от реализации алгоритмы ALS могут быть очень быстрыми.
