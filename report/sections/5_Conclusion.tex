% Заключение (еще раз повторить постановку, четко обозначить то, что было сделано и основные полученные результаты исследования)

\chapter*{Заключение}
\addcontentsline{toc}{chapter}{Заключение}

В ходе работы была рассмотрена и исследована задача неотрицательной матричной факторизации, которая формулируется следующим образом:

Для матрицы $A \in \RR^{m \times n}, A \geq 0$ и числа $k \in \mathbb{N}, k < \min\{n, m\}$
необходимо найти матрицы $W \in \RR^{m \times k}, H \in \RR^{k \times n} : W \geq 0, H \geq 0$ минимизируя функционал:
\begin{equation*}
  \min_{W \leq 0, \ H \leq 0} \dfrac{||A - WH||_F}{|| A ||_F}
\end{equation*}

Также были рассмотрены и реализованы эффективные алоритмы решения этой задачи.

Алгоритмы были успешно применены на практике для решения проблем интеллектуального анализа текста.

Таким образом, рассмотренные алгоритмы неотрицательной матричной факторизции
эффективны и надежны для извлечения и разделения статистически зависимых данных.
Тем не менее, проблема, которая все еще остается нерешённой, состоит в том,
чтобы доказать глобальную сходимость таких алгоритмов и оценить скорости сходимости данных алгоритмов.

Результаты исследование показали, что наиболее эффективным алгоритмом оказался метод попеременных наименьших квадратов на основе нормальных уравнений.
Метод выделения ключевых слов и предложений на основе НМФ был применён к тексту первой главы данной работы.
Полученные результаты хорошо резюмируют данный текст.

В будущем планируется рассмотреть новые категории алгоритмов НМФ и улучшить текущие, путём адаптации для работы с большими разреженными данными.
